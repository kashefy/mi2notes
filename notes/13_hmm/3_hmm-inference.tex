\section{Inference in HMMs: the forward-backward algorithm}

\begin{frame} 
\mode<presentation>{
    \begin{center} \huge
        \secname
    \end{center}
    }    
    \begin{center}
    What can we do with a \underline{trained} HMM model
    \end{center}
	
\end{frame}

\begin{frame}

\begin{itemize}
\item[] \textbf{Goal}: Compute the probability of a state at any step $t$ given the observed sequence:
\begin{equation}
P(\vec{m}^{(t)}  ~|~ \vec{x}^{(1)},  \ldots,\vec{x}^{(T)}) =
		P(\vec{m}^{(t)}  ~|~ \{\vec{x}^{(t-1)}\})
\end{equation}

\item[] \textbf{Assumptions}: parameters $\vec{w} := \{
		\vec{A}
		\vec{b}
		\vec{\boldsymbol{\phi}}
		\}$ are \underline{known} (have been found) We will cover estimating these parameters later.
\item[] \textbf{Procedure}: The forward-backward algorithm yields an expression for $P(\vec{m}^{(t)}  ~|~ \{\vec{x}^{(t-1)}\})$

\end{itemize}

\end{frame}

\subsection{The Forward algorithm}

\begin{frame}{\subsecname}

The Forward algorithm computes the joint distribution $P(\vec x^{(1)}, \ldots , \vec x^{(t)}, \vec m^{(t)}) \stackrel{\substack{\text{for}\\ \text{brevity}}}{=} P(\{\vec x^{(s)}\}_{s=1}^t, \vec m^{(t)})$\\
for any step $t \in \lbrack 1, T\rbrack$.\\

\svspace{5mm}

Here, $\{\vec x^{(s)}\}_{s=1}^t$ denotes the sequence of observations leading up to step $t$\\

\svspace{5mm}

Example intepretation for $P(\{\vec x^{(s)}\}_{s=1}^t, \vec m^{(t)})$: ``The distribution of temperature and humidity measures that lead up to a ``sunny'' day on day $t$.''

\end{frame}

\begin{frame}{\subsecname}

The joint distribution $P(\{\vec x^{(s)}\}_{s=1}^t, \vec m^{(t)})$ for any step $t \in \lbrack 1, T\rbrack$ is computed by \textit{marginalizing over the previous step in order to move ``forward''}:

\begin{equation}
P(\{\vec x^{(s)}\}_{s=1}^t, \vec m^{(t)})
=
\kern-3ex
\underbrace{
\sum_{\vec m^{(t-1)}}
}_{
\substack{
\text{\tiny all possible}\\
\text{\tiny valid assignments}\\
\text{\tiny at the previous}\\
\text{\tiny step t-1}
}
}
\kern-3ex
P(\{\vec x^{(s)}\}_{s=1}^t, \vec m^{(t)}, \vec m^{(t-1)})
\end{equation}

\end{frame}

\begin{frame}{\subsecname}
%\svspace{-5mm}

\begin{align}
\slidesonly{
\only<1->{
	P(\{\vec x^{(s)}\}_{s=1}^t, \vec m^{(t)})
}
\only<1,2>{
	&=
	\sum_{\vec m^{(t-1)}}
	P(\{\vec x^{(s)}\}_{s=1}^t, \vec m^{(t)}, \vec m^{(t-1)})
}
}
\only<1,2>{
	\intertext{Applying the product rule:}
	&=
	\sum_{\vec m^{(t-1)}}
	\bigg\lbrack
	P(\vec x^{(t)}~|~\{\vec x^{(s)}\}_{s=1}^{t-1}, \vec m^{(t)}, \vec m^{(t-1)})\\
	&\qquad\qquad\cdot 
	P(\vec m^{(t)}~|~\{\vec x^{(s)}\}_{s=1}^{t-1}, \vec m^{(t-1)})\\
	&\qquad\qquad\cdot 
	P(\vec m^{(t-1)},\{\vec x^{(s)}\}_{s=1}^{t-1})
	\bigg\rbrack
	}
\only<2>{
	\intertext{due to the Markov property
	$P(\vec x^{(t)}~|~\{\vec x^{(s)}\}_{s=1}^{t-1}, \vec m^{(t)}, \vec m^{(t-1)})
	=
	P(\vec x^{(t)}~|~\vec m^{(t)})
	$}\\
}
\only<3>{
	&=
	\sum_{\vec m^{(t-1)}}
	\bigg\lbrack
	P(\vec x^{(t)}~|~ \vec m^{(t)})\\
	&\qquad\qquad\cdot 
	P(\vec m^{(t)}~|~\{\vec x^{(s)}\}_{s=1}^{t-1}, \vec m^{(t-1)})\\
	&\qquad\qquad\cdot 
	\underbrace{
	P(\vec m^{(t{\color{red}-1})}, \{\vec x^{(s)}\}_{s=1}^{t{\color{red}-1}})
	}_{\text{recursion}}
	\bigg\rbrack
}
\end{align}


\only<3>{

To illustrate the recursion:

\begingroup
\footnotesize
\begin{align}
	P(&\{\vec x^{(s)}\}_{s=1}^{t{\color{red}-1}}, \vec m^{(t{\color{red}-1})})\\
	=&
	\kern-2ex
	\sum_{\vec m^{(t-1{\color{red}-1})}}
	\kern-1ex
	\bigg\lbrack
	P(\vec x^{(t{\color{red}-1})}|\vec m^{(t{\color{red}-1})})
	%\\
	%&\qquad\qquad
	%\cdot 
	P(\vec m^{(t{\color{red}-1})}|\{\vec x^{(s)}\}_{s=1}^{t-1{\color{red}-1}}, \vec m^{(t-1{\color{red}-1})})
	%\\
	%&\qquad\qquad
	%\cdot 
	\underbrace{
	P(\vec m^{(t{\color{red}-1}{\color{cyan}-1})}|\{\vec x^{(s)}\}_{s=1}^{t{\color{red}-1}{\color{cyan}-1}})
	}_{\text{recursion}}
	\bigg\rbrack
\end{align}
\endgroup
}

\end{frame}


\subsection{The Backward algorithm}

\begin{frame}{\subsecname}

The Forward algorithm computes the conditional distribution
$
P(
\underbrace{\vec x^{(t+1)}, \ldots, \vec x^{(T)}
}_{
\{\vec x^{(s)}\}_{s=t+1}^T
} | \vec m^{(t)})
$\\
for any step $t \in \lbrack0, T)$

Intepretation of the above distribution:\\

The distribution for the sub-sequence of observations \emph{following} step $t$

\slidesonly{No need to manipulate this further}

\end{frame}

\subsection{Forward-Backward}

\begin{frame}
\end{frame}
