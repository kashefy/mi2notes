\section{Inference in HMM: the forward-backward procedure}

\begin{frame} 
\mode<presentation>{
    \begin{center} \huge
        \secname
    \end{center}
    }    
    \begin{center}
    What can we do with a \underline{trained} HMM model
    \end{center}
	
\end{frame}

\begin{frame}

\begin{itemize}
\item[] \textbf{Goal}: Compute the probability of a state at any step $t$ given the observed sequence:
\begin{equation}
P(\vec{m}^{(t)}  ~|~ \vec{x}^{(1)},  \ldots,\vec{x}^{(T)}) =
		P(\vec{m}^{(t)}  ~|~ \{\vec{x}^{(t-1)}\})
\end{equation}

\item[] \textbf{Assumptions}: parameters $\vec{w} := \{
		\vec{A},
		\vec{b},
		\vec{\boldsymbol{\phi}}
		\}$ are \underline{known} (have been found) We will cover estimating these parameters later.
\item[] \textbf{Procedure}: The forward-backward procedure yields an expression for $P(\vec{m}^{(t)}  ~|~ \{\vec{x}^{(t-1)}\})$

\end{itemize}

\end{frame}

\subsection{The forward recursion}

\begin{frame}{\subsecname}

The forward procedure computes the joint distribution $P(\vec x^{(1)}, \ldots , \vec x^{(t)}, \vec m^{(t)}) \stackrel{\substack{\text{for}\\ \text{brevity}}}{=} P(\{\vec x^{(s)}\}_{s=1}^t, \vec m^{(t)})$\\
for any step $t \in \lbrack 1, T\rbrack$.\\

\svspace{5mm}

Here, $\{\vec x^{(s)}\}_{s=1}^t$ denotes the sequence of observations leading up to step $t$\\

\svspace{5mm}

Example interpretation for $P(\{\vec x^{(s)}\}_{s=1}^t, \vec m^{(t)})$: ``The distribution of temperature and humidity measures that lead up to a ``sunny'' day on day $t$.''

\end{frame}

\begin{frame}{\subsecname}

The joint distribution $P(\{\vec x^{(s)}\}_{s=1}^t, \vec m^{(t)})$ for any step $t \in \lbrack 1, T\rbrack$ is computed by \textit{marginalizing over the previous step in order to move ``forward''}:

\begin{equation}
P(\{\vec x^{(s)}\}_{s=1}^t, \vec m^{(t)})
=
\kern-3ex
\underbrace{
\sum_{\vec m^{(t-1)}}
}_{
\substack{
\text{\tiny all possible}\\
\text{\tiny valid assignments}\\
\text{\tiny at the previous}\\
\text{\tiny step t-1}
}
}
\kern-3ex
P(\{\vec x^{(s)}\}_{s=1}^t, \vec m^{(t)}, \vec m^{(t-1)})
\end{equation}

\end{frame}

\begin{frame}{\subsecname}
%\svspace{-5mm}

\begin{align}
\slidesonly{
\only<1->{
	P(\{\vec x^{(s)}\}_{s=1}^t, \vec m^{(t)})
}
\only<1,2>{
	&=
	\sum_{\vec m^{(t-1)}}
	P(\{\vec x^{(s)}\}_{s=1}^t, \vec m^{(t)}, \vec m^{(t-1)})
}
}
\only<1,2>{
	\intertext{Applying the product rule:}
	&=
	\sum_{\vec m^{(t-1)}}
	\bigg\lbrack
	P(\vec x^{(t)}~|~\{\vec x^{(s)}\}_{s=1}^{t-1}, \vec m^{(t)}, \vec m^{(t-1)})\\
	&\qquad\qquad\cdot 
	P(\vec m^{(t)}~|~\{\vec x^{(s)}\}_{s=1}^{t-1}, \vec m^{(t-1)})\\
	&\qquad\qquad\cdot 
	P(\vec m^{(t-1)},\{\vec x^{(s)}\}_{s=1}^{t-1})
	\bigg\rbrack
	}
\only<2>{
	\intertext{due to the Markov property
	$P(\vec x^{(t)}~|~\{\vec x^{(s)}\}_{s=1}^{t-1}, \vec m^{(t)}, \vec m^{(t-1)})
	=
	P(\vec x^{(t)}~|~\vec m^{(t)})
	$}\\
}
\only<3>{
	&=
	\sum_{\vec m^{(t-1)}}
	\bigg\lbrack
	P(\vec x^{(t)}~|~ \vec m^{(t)})\\
	&\qquad\qquad\cdot 
	P(\vec m^{(t)}~|~\{\vec x^{(s)}\}_{s=1}^{t-1}, \vec m^{(t-1)})\\
	&\qquad\qquad\cdot 
	\underbrace{
	P(\vec m^{(t{\color{red}-1})}, \{\vec x^{(s)}\}_{s=1}^{t{\color{red}-1}})
	}_{\text{recursion}}
	\bigg\rbrack
}
\end{align}


\only<3>{

To illustrate the recursion:

\begingroup
\footnotesize
\begin{align}
	P(&\{\vec x^{(s)}\}_{s=1}^{t{\color{red}-1}}, \vec m^{(t{\color{red}-1})})\\
	=&
	\kern-2ex
	\sum_{\vec m^{(t-1{\color{red}-1})}}
	\kern-1ex
	\bigg\lbrack
	P(\vec x^{(t{\color{red}-1})}|\vec m^{(t{\color{red}-1})})
	%\\
	%&\qquad\qquad
	%\cdot 
	P(\vec m^{(t{\color{red}-1})}|\{\vec x^{(s)}\}_{s=1}^{t-1{\color{red}-1}}, \vec m^{(t-1{\color{red}-1})})
	%\\
	%&\qquad\qquad
	%\cdot 
	\underbrace{
	P(\vec m^{(t{\color{red}-1}{\color{cyan}-1})}|\{\vec x^{(s)}\}_{s=1}^{t{\color{red}-1}{\color{cyan}-1}})
	}_{\text{recursion}}
	\bigg\rbrack
\end{align}
\endgroup
}

\end{frame}


\subsection{The backward recursion}

\begin{frame}{\subsecname}

The backward recursion computes the conditional distribution:
\begin{equation}
P(
\underbrace{\vec x^{(t+1)}, \ldots, \vec x^{(T)}
}_{
\{\vec x^{(s)}\}_{s=t+1}^T
} | \vec m^{(t)})
\qquad 
\text{for any step }t \in \lbrack0, T)
\end{equation}

Interpretation of the above\notesonly{ distribution}:\\

The distribution for the sub-sequence of observations \emph{following} step $t$

\slidesonly{No need to manipulate this further}

\pause

\question{Where is the recursion?}

\pause

\svspace{-5mm}

\begin{align}
P(&\vec{x}^{(t+1)}, \ldots, \vec{x}^{(T)} ~|~ \vec{m}^{(t)}) =\\
				&\sum_{\big\{\vec{m}^{(t{\color{red}+1})}\big\}} \kern-2ex
				\underbrace{
				P(\vec{x}^{(t+1{\color{red}+1})}, \ldots, \vec{x}^{(T)} | \vec{m}^{(t{\color{red}+1})})
				}_{
				P(\vec{x}^{(t{\color{red}+2})}, \ldots, \vec{x}^{(T)} | \vec{m}^{(t{\color{red}+1})})
				}
				%{\color{orange}\beta(\vec{m}^{(t+1)})} ~
				P(\vec{x}^{(t{+1})} | \vec{m}^{(t{\color{red}+1})}) ~
				P(\vec{m}^{(t{\color{red}+1})} | \vec{m}^{(t)})
\end{align}


\end{frame}

\subsection{The forward-backward procedure}

\begin{frame}{\subsecname: Putting things together}

\begin{equation}
P(
\vec m^{(t)} | \{\vec x^{(s)}\}_{s=1}^T )
\propto 
P(\vec m^{(t)} , \{\vec x^{(s)}\}_{s=1}^T)
\end{equation}

Here, $\{\vec x^{(s)}\}_{s=1}^T$ is the entire observed sequence.

\pause

\begin{align}
P(\vec m^{(t)} , \{\vec x^{(s)}\}_{s=1}^T)
&= 
P( \{\vec x^{(s)}\}_{s=1}^T, \vec m^{(t)})\\
\stackrel{\substack{\text{product}\\\text{rule}}}{=}&
P( \vec x^{(t+1)}, \ldots, \vec x^{(T)} | \vec m^{(t)}, \vec x^{(1)}, \ldots, \vec x^{(t)})\\
&\qquad\cdot 
P(\vec m^{(t)}, \vec x^{(1)}, \ldots, \vec x^{(t)}) 
\end{align}

\pause

We can simplify the above by exploiting the Markov property.\\
Specifically, that
$\vec x^{(t+1)}, \ldots, \vec x^{(T)}$ are conditionally independent of $\vec x^{(1)}, \ldots, \vec x^{(t)}$ \underline{given} $\vec m^{(t)}$.



\end{frame}

\begin{frame}{\subsecname: Putting things together}

\notesonly{From this follows:}

\slidesonly{
\only<1>{
\begin{align}
P(\vec m^{(t)} , \{\vec x^{(s)}\}_{s=1}^T)
&= 
P( \vec x^{(t+1)}, \ldots, \vec x^{(T)} | \vec m^{(t)} )
\cdot 
P(\vec m^{(t)}, \vec x^{(1)}, \ldots, \vec x^{(t)}) 
\end{align}
}
}
\only<2>{
\begin{align}
P(\vec m^{(t)} , \{\vec x^{(s)}\}_{s=1}^T)
&= 
\underbrace{\color{orange}
P( \vec x^{(t+1)}, \ldots, \vec x^{(T)} | \vec m^{(t)} )
}_{
\substack{
\text{from backward}\\
\text{recursion}
}
}
\cdot 
\underbrace{\color{blue}
P(\vec m^{(t)}, \vec x^{(1)}, \ldots, \vec x^{(t)}) 
}_{
\substack{
\text{from forward}\\
\text{recursion}
}
}
\end{align}
}

\pause

\begin{align}
P(
\vec m^{(t)} | \{\vec x^{(s)}\}_{s=1}^T )
\; \propto& \;
{\color{orange}
P( \vec x^{(t+1)}, \ldots, \vec x^{(T)} | \vec m^{(t)} )
}
\cdot 
{\color{blue}
P(\vec m^{(t)}, \vec x^{(1)}, \ldots, \vec x^{(t)}) 
}\\[5mm]
=& \;
\frac{
{\color{orange}
P( \vec x^{(t+1)}, \ldots, \vec x^{(T)} | \vec m^{(t)} )
}
\cdot 
{\color{blue}
P(\vec m^{(t)}, \vec x^{(1)}, \ldots, \vec x^{(t)}) 
}
}{
P(\vec x^{(1)}, \ldots, \vec x^{(T)})
}
\label{eq:hmmcondfwdbwd}
\end{align}

\pause

\question{Where did the normalization factor come from?}

\pause

\notesonly{-The normalization in \eqref{eq:hmmcondfwdbwd} comes from Bayes' theorem:
}
\begin{equation}
P(\vec{m}^{(t)}  ~|~ \{\vec{x}^{(s)}\})
\kern-2ex
	\underbrace{=}_{{\tiny \text{Bayes theorem}}} 
\kern-3ex
	\frac{P( \{\vec{x}^{(s)}\} ~|~ \vec{m}^{(t)})
	\cdot
	P(\vec{m}^{(t)})}
		 {P( \{\vec{x}^{(s)}\})}
\end{equation}

\end{frame}


