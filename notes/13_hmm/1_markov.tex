\section{Markov chains}

\begin{frame} 
\mode<presentation>{
    \begin{center} \huge
        \secname
    \end{center}
    }
\mode<presentation>{
    \begin{center}
    Remember them from stochastic optimization?
    \end{center}
    }
\end{frame}

\begin{frame}{\secname}

Consider the random variables $y^{(1)}, y^{(2)}, \ldots, y^{(T-1)}, y^{(T)}$.
There is \textbf{no} statistical independence between the $y$'s:
\begin{equation}
P(y^{(1)}, y^{(2)}, \ldots, y^{(T-1)}, y^{(T)}) = \prod_{t=1}^T P(y^{(t)})
\end{equation}

But

\begin{equation}
P(y^{(t)} | y^{(t-1)}, y^{(t-2)}, \ldots, y^{(2)}, y^{(1)}) \ne P(y^{(t)} | y^{(t-1)})
\end{equation}

$y{(t)}$ depends only on $y^{(t-1)}$ $\rightarrow$ \emph{Markov property}

A sequence of samples of these $y$'s $\rightarrow$ \emph{Markov chain}

\end{frame}
