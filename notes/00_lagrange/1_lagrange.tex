\section{Precursor: Method of Lagrange Multipliers}

\definecolor{darkgreen}{rgb}{0,0.6,0}

\begin{frame}
	\slidesonly{
	\begin{center}\Large
	\secname
	\end{center}
	\pause
	}
    \begin{center}
    \slidesonly{\huge}
	Don't panic!
    \end{center}
    \begin{center}
        Essentially just hill-climbing (gradient ascent)
    \end{center}
    \pause

\underline{Objective}: \\
\textit{Maximize} an objective function while also satisfying some constraint(s).\\
{
\small(\textit{Maximize}: Find the arguments that maximize an objective function.)
}\\

Don't forget: any maximization problem can be turned into a minimization problem\notesonly{ by maximizing the ``negative'' of the function}.
\end{frame}

\subsection{Gradient ascent - unconstrained optimization}
\begin{frame}\frametitle{\subsecname}
Maximize an objective function $f(w_1, w_2)$ (no constraints here).\\
Here $w_1$ and $w_2$ are referred to as free parameters.\\

\begin{figure}[h]
	\centering
	\includegraphics[width=0.8\linewidth]{img/lagrange_objfunction}%
	\notesonly{
	\caption{
	An unconstrained optimization problem.
	}%
	}%
    \label{fig:unconstrained}%
\end{figure}

\mode<article>{
\figref{fig:unconstrained} depicts the objective function $f$ we want to maximize (i.e. reach the blue $\color{blue}\times$ at the top). $c_1$ and $c_2$ are level curves where $f$ is constant. 
	The blue arrows indicate the direction of the gradient $\vec \nabla f$, which is always perpendicular to the level curve.
}

\slidesonly{
\vspace{-3mm}
}

The gradient $\vec \nabla f$ describes the direction of greatest ascent:\\
\slidesonly{
\vspace{-3mm}
}
\begin{equation}
\vec \nabla f = 
\frac{\partial f}{\partial \vec w} = 
\rmat{\frac{\partial f}{\partial w_1} \\[0.2cm] \frac{\partial f}{\partial w_2} }
\end{equation}

\end{frame}

\subsection{Adding a single constraint}

\begin{frame}\frametitle{\subsecname}
We restrict solutions to those that satisfy the constraint $g(w_1, w_2) = c$:

\begin{figure}[h]
\centering
\includegraphics[width=0.7\linewidth]{img/lagrange_objfunction_constrained}
\mode<article>{
\caption{The surface of the constraint cuts through the surface of the objective function. The highest point that can be reached on $f$ which also intersects with $g$ is unique in that the gradients of $f$ and $g$ both point in the same direction with only a difference in scale.}
}
\end{figure}

\only<1,2>{
\question{What is characteristic of the solution ${\color{darkgreen}\vec w^{*}}$ to the constrained optimization problem?}\\
}
\notesonly{
-The solution of the constrained optimization problem is characterized by:
}
\only<2>{
\slidesonly{\vspace{-5mm}}
\begin{equation}
    \vec \nabla f \big|_{\color{darkgreen}\vec w^{*}} = \lambda' \vec \nabla g\big|_{\color{darkgreen}\vec w^{*}} \,,\quad
\text{where $\lambda'$ is a scaling factor.}
\label{eq:equality}
\end{equation}
}
\notesonly{\eqref{eq:equality} holds for the highest position in $f$ while also satisfying the constraint.\\

Consequently,}
% temporarily change footnote marks to symbols so not to confuse with exponents
\renewcommand*{\thefootnote}{\fnsymbol{footnote}}
\only<3>{
\slidesonly{\vspace{-5mm}}
\begin{align}
  \vec \nabla f \; - \lambda' \vec \nabla g     &= \vec 0 \\
  \vec \nabla f \; + \underbrace{(- \lambda')}_{=:\lambda} \vec \nabla g &= \vec 0 \\
  \vec \nabla f \;  + \lambda \vec \nabla g      &= \vec 0\notesonly{\;\footnotemark }
  \label{eq:equalitylambda}
\end{align}
}
\notesonly{
    \footnotetext{
    The switch from $\lambda'$ to $\lambda$ is to be more consistent with the lecture slides.
    }
}
% change footnote marks back to original scheme (numbers)
\renewcommand*{\thefootnote}{\arabic{footnote}}

\end{frame}

\begin{frame}
\only<1,2>{
The constrained optimization problem is formulated as:
\begin{equation}
\underbrace{f(w_1, w_2) \;\eqexcl\; \max_{w_{1},w_{2}}}_{\text{maximization}} \quad  \text{subject to} \quad \underbrace{g(w_1,w_2)\;-c\; = \; 0}_{\text{a single constraint}}
\label{eq:optconstrained}
\end{equation}
}
\notesonly{
The \emph{Lagrangian} function reformulates the constrained optimization problem from \eqref{eq:optconstrained} in a way that reflects the relationship of the two gradients at the solution and thus facilitate finding the solution \eqref{eq:equalitylambda}.} The \emph{Lagrangian} is \notesonly{therefore }defined as:

\only<2,3>{
\begin{equation}
L(w_1, w_2, \lambda) \; := \; f(w_1,w_2) + \lambda\big(g(w_1, w_2)-c\big)
\end{equation}

Setting the gradient $\nabla L$ to zero guarantees a solution at which $\vec \nabla f \;  + \lambda \vec \nabla g = \vec 0$
}
\only<3>{

\begin{equation}
\vec \nabla L = 
\rmat{
	\frac{\partial L}{\partial w_1} \\[0.5cm]
	\frac{\partial L}{\partial w_2} \\[0.5cm]
	\frac{\partial L}{\partial \lambda}
	}
=
\rmat{
	\;\frac{\partial f}{\partial w_1} \quad\;\;+\quad\;\; \lambda \frac{\partial g}{\partial w_1} \;\\[0.3cm]
	\;\frac{\partial f}{\partial w_2} \quad\;\;+\quad\;\; \lambda \frac{\partial g}{\partial w_2} \;\\[0.3cm]
	\underbrace{\frac{\partial f}{\partial \lambda}}_{=0} \;+\quad g(w_1, w_2)-c
	}
=
\rmat{
	0 \\[0.5cm]
	0 \\[0.5cm]
    0
	}
= \vec 0
\label{eq:lagrangiangrad}
\end{equation}

Setting the first two elements of $\vec \nabla L$\notesonly{, namely $\frac{\partial L}{\partial w_1}$ and $\frac{\partial L}{\partial w_2}$,} to zero ensures that $\nabla f = -\lambda \nabla g$,\\
while $\frac{\partial L}{\partial \lambda}=0$ ensures that $g(w_1, w_2) = c$.\\

\notesonly{\eqref{eq:lagrangiangrad} describes a system of }3 equations with 3 unknowns.\notesonly{
\footnote{
Not much emphasis is put on knowing how to solve this by hand. If interested, the Wikipedia article on \href{https://en.wikipedia.org/wiki/Lagrange_multiplier\#Examples}{Lagrange multiplier} provides some examples.}
}\\

We refer to $\lambda$ as the \emph{multiplier} for the constraint $c$.
}
\end{frame}

\newpage

\subsection{Multiple constraints}

\begin{frame}
 
\question{How do we extend this to multiple {\color{red}{equality}} constraints?}

\slidesonly{\vspace{-2mm}}
    
\begin{equation}
\underbrace{f_0(\vec w) \;\eqexcl\; \text{max}}_{\text{maximization}} \quad  \text{and} \quad f_k(\vec w)\;{\color{red}{=}}\; \; 0 \;, \quad k = 1,\ldots,m
\label{eq:optimizationequalitymultipe}
\end{equation}

where $m$ denotes the number of constraints. $f_{0}(\vec w)$ is reserved for the function to be optimized., while $f_{1}(\vec w), f_{2}(\vec w),\ldots,f_{m}(\vec w)$ is used for all $m$ constraints.

The Lagrangian for multiple constraints is defined as:
%\slidesonly{\vspace{-2mm}}
\begin{align}
L(\,\vec w\;, \overbrace{\{\lambda_k\}}^{
\mathclap{
\substack{\text{a multiplier} \\
\text{for each constraint}}
}
}) 
\; :=& \; 
f_0(\vec w) + \lambda_1 \, f_1(\vec w) + \lambda_2 \, f_2(\vec w) + \, \ldots \, + \lambda_m \, f_m(\vec w) \\
\; =& \; 
f_0(\vec w) + \sum_{k=1}^{m} \lambda_k \, f_k(\vec w)
\label{eq:lagrangianmultiple}
\end{align}

\question{What if we have {inequality} constraints?}

\end{frame}

\begin{frame}

In the case of {\color{red}{inequality}} constraints, our constrained optimization problem would have the following form:

\begin{equation}
\underbrace{f_0(\vec w) \;\eqexcl\; \text{max}}_{\text{maximization}} \quad  \text{and} \quad f_k(\vec w)\; {\color{red}{\le}} \; 0 \;, \quad k = 1,\ldots,m{}
\label{eq:optimizationINequalitymultipe}
\end{equation}

This leads to the following changes to Lagrangian in order to take the inequality and the type of optimization (\textcolor{magenta}{max} vs. min) into account. The inequality is taken into account in that the solutions for $\lambda$ extend to some range:

\begin{equation}
L(\,\vec w\;, \{\lambda_k\}
) \; := \; {\color{magenta}-} \; f_0(\vec w) + \sum_{k=1}^{m} \lambda_k \, f_k(\vec w)\,,\qquad
\lambda_{k} \;{\color{red}{\ge}}\; 0 \quad \forall k \in \{1,\ldots,m\}
\label{eq:lagrangianINequalitymultiple}
\end{equation}
 
    
\end{frame}
