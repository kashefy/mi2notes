\subsection{Whitening in the context of ICA}

\mode<presentation>{
\begin{frame} 
    \begin{center} \huge
        \subsecname
    \end{center}
    \begin{center}
    Bringing PCA into the mix.
    How PCA solves half of the ICA problem.
    \end{center}
\end{frame}
}

\begin{frame}{\subsecname}

\slidesonly{
We've established that
\begin{equation}
\vec \Sigma_x =  \vec A\, \vec A^\top
\end{equation}
}

\pause

\notesonly{
If we were to apply \emph{Singular Value Decomposition} on the symmetric matrix $\vec \Sigma_x$:
}

\begin{equation}
\mathrm{SVD}(\vec \Sigma_x) = \vec U\, \vec D \, \vec U^\top
\end{equation}

where\\
\notesonly{
$\vec U$ is an orthogonal matrix,} $\vec U^\top\vec U = \vec I_N$ and
\notesonly{$\vec D$ is a diagonal matrix with rank $N$, }$\vec D^\top = \vec D$.\\

We define the two following transformation $\vec Q$ and $\widetilde{\vec A}$:
\begin{equation}
    \label{eq:defq}
    \vec Q := \vec D^{-\frac{1}{2}} \vec U^\top
    \end{equation}
    \begin{equation}
    \label{eq:defatilde}
    \widetilde{\vec A} := \vec Q \, \vec A 
\end{equation}

Which yields\only<3>{ a new ICA problem:
\begin{equation}
\vec u = \vec Q\, \vec x = \vec Q\,\vec A \, \vec s = \widetilde{\vec A} \, \vec s
\end{equation}
}

\end{frame}

\begin{frame}{A new ICA problem}

\slidesonly{
\begin{equation}
\vec u := \vec Q\, \vec x = \vec Q\,\vec A \, \vec s = \widetilde{\vec A} \, \vec s
\end{equation}
}

\question{What is so special about the new mixing matrix $\widetilde{\vec A}$?} \\

\notesonly{-$\widetilde{\vec A}$ is orthogonal (i.e. $\widetilde{\vec A}\, \widetilde{\vec A}^\top = \widetilde{\vec A}^\top \widetilde{\vec A} = \widetilde{\vec A}^{-1} \widetilde{\vec A} = \vec I_N$).
 We will see the implications of this in how it reduces the number of free parameters for solving the new ICA problem.
}

\pause

\slidesonly{
\begin{center}
	\includegraphics[width=0.4\textwidth]{img/meme_newalwaysbetter}%
\end{center}

- $\widetilde{\vec A}$ is orthogonal. Not necessarily the case for the original mixing matrix $\vec A$.
}

\end{frame}

\begin{frame}{A new ICA problem}

\slidesonly{
\vspace{-14mm}
\hspace{8.0cm}
\StickyNote[1.7cm]{
	\begingroup
	\scriptsize
	%\begin{equation}
		%\widehat{P}_{u_i} (\widehat{u}_i) =
		 %\frac{1}{\big| \widehat{f}_i^{'} (\widehat{s}_i) \big|} 
		 %\widehat{P}_{s_i}(\widehat{s}_i)
	%\end{equation}
	\begin{equation}
		\vec Q := \vec D^{-\frac{1}{2}} \vec U^\top
	\end{equation}
\vspace{-2mm}
	\begin{equation}
	\vec \Sigma_x = \E \lbrack\, \vec x \, \vec x^\top \rbrack = \vec U \, \vec D \, \vec U^{\top}
	\end{equation}
	\endgroup
}[3.3cm] % width
\vspace{-22mm}
}

\begin{align}
\E \lbrack\, \vec u \, \vec u^\top \rbrack
&\mystackrel{\vec u := \vec Q\, \vec x}{\vec u = \vec Q\, \vec x}
 \E \lbrack\, \vec Q \, \vec x  \left( \vec Q \, \vec x\right)^\top \rbrack\\
&\mystackrel{\vec u = \vec Q\, \vec x}{}\E \lbrack\, \vec Q \, \vec x \; \vec x^\top \vec Q^\top \rbrack\\
&\mystackrel{\vec u = \vec Q\, \vec x}{} \E \lbrack\, \vec Q \, \vec \Sigma_{x} \, \vec Q^\top \rbrack\\
&\mystackrel{\vec u = \vec Q\, \vec x}{} \vec Q \, \vec \Sigma_{x} \, \vec Q^\top\\
&\mystackrel{\vec u = \vec Q\, \vec x}{}
\left( \vec D^{-\frac{1}{2}} \vec U^\top \right)
\left( \vec U \, \vec D \, \vec U^\top \right)
\left( \vec D^{-\frac{1}{2}} \vec U^\top \right)^\top
\end{align}

\end{frame}
\begin{frame}

\slidesonly{
\begin{align}
\E \lbrack\, \vec u \, \vec u^\top \rbrack
&\mystackrel{\vec u = \vec Q\, \vec x}{}
\left( \vec D^{-\frac{1}{2}} \vec U^\top \right)
\left( \vec U \, \vec D \, \vec U^\top \right)
\left( \vec D^{-\frac{1}{2}} \vec U^\top \right)^\top
\end{align}
}

From $\vec U^\top\vec U = \vec I_N$ and $\vec D^\top = \vec D$ follows:

\begin{align}
\E \lbrack\, \vec u \, \vec u^\top \rbrack
&=
\vec D^{-\frac{1}{2}}
\underbrace{\left( \vec U^\top  \vec U \right)}_{= \vec I_N}
\, \vec D \, 
\underbrace{\left( \vec U^\top \,  \vec U \right)}_{= \vec I_N}
\,
\left(\vec D^{-\frac{1}{2}} \right)^\top \\
&= \vec D^{\top}
\, 
\underbrace{\vec D^{-\frac{1}{2}}  \, 
\left(\vec D^{-\frac{1}{2}} \right)^{\cancel{\top}}}_{=\vec D^{-1}} = {\color{blue}\vec I_N}
\end{align}


Recall 
\notesonly{from \eqref{eq:sigmax}:}
\begin{equation}
\vec \Sigma_x =  \E \lbrack \, \vec x \, \vec x^\top \rbrack
=  \vec A\, \vec A^\top
\end{equation}
Therefore:
\begin{equation}
\E \lbrack \, \vec u \, \vec u^\top \rbrack
= \widetilde {\vec A}\, \widetilde {\vec A}^\top = {\color{blue}\vec I_N}
\end{equation}

\notesonly{
The new mixing matrix}
$\widetilde{\vec A}$ is orthonormal.\\

\end{frame}

\begin{frame}{\subsecname}

\slidesonly{

What makes $\widetilde{\vec A}$ so special is that it is orthonormal.
}

\pause

\svspace{5mm}

The space of solutions restricted to finding unmixing matrices that are also orthogonal.
This restriction can speed up convergence of ICA. 
The number of free parameters for solving the new ICA problem 
$\vec u = \widetilde{\vec A}\, \vec s$
is reduced\notesonly{ from $N^2$ to $N(N-1)/2$
}\slidesonly{:\\
$$N^2 \rightarrow N(N-1)/2$$
}

\end{frame}

\begin{frame}
\slidesonly{
\frametitle{What just happened?}

\begin{itemize}
\setlength\itemsep{1em}
\item We've transformed $\vec x$ to get $\vec u$:
$ \qquad \qquad \qquad
\vec u := \vec D^{-\frac{1}{2}} \vec U^\top \vec x
$

\item Where did $\vec D$ and $\vec U$ come from?
$ \qquad \qquad
\mathrm{SVD}(\vec \Sigma_x) = \vec U\, \vec D \, \vec U^\top
$

\item What do $\vec D$ and $\vec U$ represent? - the eigenval. and eigenvect. of $\vec \Sigma_x$

\item What's so special about $\vec u$?
$ \qquad \qquad \qquad \qquad
\E \lbrack \, \vec u \, \vec u^\top \rbrack = \vec I_N
$

\pause

\item ...So?

\pause

\begin{itemize}
\item[-] new ICA problem: $\vec u := \widetilde{\vec A}\, \vec s$\\
\item[-] $\widetilde{\vec A}$ is orthogonal\\
\item[-] unmixing the new problem involves only ``half'' the number of free parameters.
\end{itemize}

\question{What is the transformation $\vec D^{-\frac{1}{2}} \vec U^\top$ called?}

\notesonly{- It's a whitening transformation. The same we know from PCA.}

\end{itemize}
}

\end{frame}

\subsubsection{Whitening solves half of the ICA problem}

\begin{frame}{\subsubsecname}

\slidesonly{

Unmixing the new problem involves only ``half'' the number of free parameters.
$$N^2 \rightarrow N(N-1)/2$$
}

\pause

\svspace{-5mm}

\question{Does this mean that $\vec W$ is no longer $N \times N$?}

\pause

-No $\vec W$ is still $N \times N$, but we no longer have to search for each component individually.

\end{frame}

\begin{frame}{\subsubsecname}

Example with $N=2$ uniformly distributed sources $s_1, s_2$:

 \begin{center}
  \begin{minipage}{0.29\textwidth}
\includegraphics[width=0.99\textwidth]{./img/uniform_original_sources} 
  \end{minipage}
  \hspace{5mm}
  \begin{minipage}{0.29\textwidth}
\includegraphics[width=0.99\textwidth]{./img/uniform_mixtures_centered} 
  \end{minipage}\\
  
  \begin{minipage}{0.29\textwidth}
\includegraphics[width=0.99\textwidth]{./img/uniform_mixtures_decorrelated} 
  \end{minipage}
  \hspace{5mm}
  \begin{minipage}{0.29\textwidth}
\includegraphics[width=0.99\textwidth]{./img/uniform_mixtures_whitened} 
  \end{minipage}
  \notesonly{
  \captionof{figure}{Example with $N=2$ uniformly distributed sources $s_1, s_2$}
  }
\end{center}

\end{frame}

\begin{frame}{\subsubsecname}

\slidesonly{
Example with $N=2$ uniformly distributed sources $s_1, s_2$:

 \begin{center}
  \begin{minipage}{0.25\textwidth}
\includegraphics[width=0.99\textwidth]{./img/uniform_original_sources} 
  \end{minipage}
  \hspace{5mm}
  \begin{minipage}{0.25\textwidth}
\includegraphics[width=0.99\textwidth]{./img/uniform_mixtures_whitened} 
  \end{minipage}
\end{center}
}

ICA on whitened mixtures is about ``rotating it back''.

2D data can be rotated by an angle $\theta$ using:

\begin{align}
  \vec{x}_{\theta} & = 
  \begin{pmatrix}
	\cos(\theta) & -\sin(\theta) \\
	\sin(\theta) & \cos(\theta)
  \end{pmatrix}
  \vec{x} \qquad \text{with} \quad \theta = 0, \, \ldots\, , 2\pi
\end{align}

\pause

Find the right rotation for \textcolor{blue}{2}D data involves a \textcolor{red}{single} free parameter.
\begin{equation}
\# \text{free parameters} = \textcolor{blue}{N} \cdot (\textcolor{blue}{N}-1)/2 = \textcolor{blue}{2} \cdot (\textcolor{blue}{2}-1)/2 = \textcolor{red}{1}
\end{equation}

\end{frame}

\slidesonly{
\begin{frame}{Summary so far}
\begin{enumerate}
\item Initial ICA Problem: $\vec x = \vec A\, \vec s$
\item New ICA Problem: $\vec u := \widetilde{\vec A}\, \vec s$,\\

where $\vec u = \vec D^{-\frac{1}{2}} \vec U^\top \vec x$ and $\vec \Sigma_u = \vec I_N$.
\item $\vec u$ is the \emph{whitened} version of $\vec x$.
\item $\vec D$ and $\vec U$ can be obtained via PCA on $\vec x$.
\item Applying ICA on whitened data reduces the number of free parameters.
\item PCA simplifies the ICA problem.
\item ICA on whitened data is about ``rotating it back''.
\end{enumerate}

\end{frame}
}

