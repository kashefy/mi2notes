\section{Mixture Models}


\begin{frame} 
\mode<presentation>{
    \begin{center} \huge
        \secname
    \end{center}
    }
    \begin{center} \large
        A class of models that can cluster the data and keep track of importance by estimating densities around clusters.
    \end{center}
	
\end{frame}

\subsection{Many ways of describing the same thing}

\begin{frame}{\subsecname}

Mixture Models\ldots
\begin{itemize}
\item combine density estimation and clustering
\item cluster the data and estimate the density around each cluster
\item increase the resolution of your density estimation in order to end up with a mode around each cluster of data
\item Assume $M$ source are generating points in the data. Find a measure for:
	\begin{enumerate}
	\item the probability of a source $\{q_1,q_2,\ldots,q_M\}$ generating \emph{any} point (How much are you contributing to the data?)
	\item the probability of a point coming from source $q_1$ vs. $q_2$ vs. \ldots vs. $q_M$. Who generated this point?
	\end{enumerate}
\end{itemize}

\end{frame}

\begin{frame}{Model class}

Let $\vec x \in \R^N \sim P(\vec x)$ (data generating distribution)

The Mixture Model breaks down the data generating distribution $P(\vec x)$ into\notesonly{ a linear combination as follows}:

\begin{equation}
	P_{(\vec{x})} = \sum_{q=1}^{M} P_{(\vec{x} | q)} P_{(q)}
\end{equation}
	
$P_{(\vec{x} | q)}:$ components: probability density, that data point $\vec{x}$ was created by component $q$.
\\\vspace{0.3cm}
$P_{(q)}:$ mixture parameters: probability, that component $q$ creates a data point. (How much are you contributing to the data?)

\end{frame}
